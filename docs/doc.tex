\documentclass[10pt]{article}

\begin{document}
\title{Dynamics User Documentation}
\maketitle


\section*{Introduction}
\subsection*{Motivation}
This code is aiming for solving nonliner dynamic problems using numerical method,
which is inspired by a book, "Nonliner Dynamics and Chaos" by Steven H. Strogatz.
This repository provides a tool to understand nonlinear model visually, which allows
users to have a greater understanding of nonlinear dynamics.

\subsection*{User side code}
The user side codes consist of the following: \medskip \newline
\textbf{dof:} The degree of freedom of the nonlinear model. \medskip \newline
\textbf{time step:} The numerical increment of the nonlinear model. \medskip \newline
\textbf{reference point:} The reference position of model.

\subsection*{Requirement}
Dynamics utilises:
\begin{enumerate}
    \item \textbf{numpy} and \textbf{scipy} for numerical calculation and 
    operation.
    \item \textbf{pyqt5}, \textbf{pyqtgraph} and \textbf{PyOpenGL} for visualisation
    of the data.
  \end{enumerate}

\section*{Framework}
Dynamics consists of four major components, \textbf{Problem}, \textbf{DynamicModel},
\textbf{Component} \& \textbf{Solver}.
\subsection*{Problem}
is 
\subsection*{Dynamic Model}
amazing
\subsection*{Component}
and
\subsection*{Solver}
stupid

\section*{Pendulum}

\section*{Double pendulum}

\end{document}


